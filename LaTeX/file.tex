\documentclass[10pt]{beamer}

\usetheme[progressbar=frametitle]{metropolis}
\usepackage{appendixnumberbeamer}

\usepackage{booktabs}
\usepackage[scale=2]{ccicons}

\usepackage{pgfplots}
\usepgfplotslibrary{dateplot}
\usepackage{multicol}

\usepackage{xspace}
\usepackage{xcolor}
\setbeamercolor{alerted text}{fg=red}

\title{EEE212 Industrial Awareness and Group Project}
\subtitle{Smart Refrigrator}
\date{\today}
%\date{}
\author{Rishabh Aggarwal, 1407014, ....}
\institute{Department of Electrical and Electronics Engineering, XJTLU}
% \titlegraphic{\hfill\includegraphics[height=1.5cm]{logo.pdf}}

\begin{document}

\maketitle
{
\begin{frame}{Table of contents}
  \setbeamertemplate{section in toc}[sections numbered]
  \tableofcontents[hideallsubsections]
\end{frame}
}
\section{Introduction}
{
\begin{frame}[fragile]{Technical Writing}

  T
  
\end{frame}
}

\section{Major Equipment Utilized}
{
\begin{frame}{The 12 issues!}
	The following were the major components of the refrigerator
	\begin{enumerate}[<+- | alert@+>]
		\item Peltier Module
		\item Arduino
		\item Relay
		\item Temperature Probe/Sensor
		\item Plexiglass
		\item Power Supply
		\item Foam Sheets
	\end{enumerate}
\end{frame}
}
\section{How to overcome them?}
{
\begin{frame}{1. What is Poor Organization}
	It all comes down to:\\
	\textit{"If the reader believes the content has some important to him, he can plow through a report even if it is dull or has lengthy sentences and big words. But, if it is poorly organized --- forget it. There is no wat to make sense of what is written."}
\end{frame}
}

{
	\begin{frame}{Causes for Poor Organization}
		Poor organization stems from poor planning. Not knowing what and how to write results in a paper that is inconsistent and hard to follow. These make it non-technical. 
	\end{frame}
}

{
\begin{frame}{Improving Organization}
	A builder requires detailed blueprints of a building before beginning to construct it. Likewise, engineering paper needs a rough outline that highlights the contents of the paper. 

	\begin{alertblock}{Having an outline}
		An outline helps break down the writing project into smaller pieces, in order to simply the task. It is a tool to aid in organization of ideas. 
	\end{alertblock}
\end{frame}
}

{
\begin{frame}{Standard Outlines}
	Standard formats are best suited when writing a technical paper. These differ depending on the paper. For example, how a laboratory report differs from an operating manual. 

	\begin{alertblock}{Non-standard formats}
		If the format isn't strictly defined, select the organizational scheme that best suits the material. 
		These are:
		\begin{multicols}{2}
		\begin{itemize}
			\item Order based on location
			\item Order based on difficulty 
			\item Alphabetical order
			\item Chronological order
			\item Problem/solution
			\item Inverted Pyramid
			\item Deductive order
			\item Inductive order
			\item List
		\end{itemize}
		\end{multicols}
	\end{alertblock}
\end{frame}
}

{
	\begin{frame}{2. What does it mean by Misreading the Reader}
		Any communication, written or spoken, targeting an unintended audience or not addressed to a specific audience properly. \\
		\vspace{5mm}
		Technical reports must therefore be written around the needs, interests and desires of the reader. Written communications are most effective when they are targeted and personal.
	\end{frame}
}
{
	\begin{frame}{Developing an image of the reader}
		For technical papers, you are targeting many readers, not an individual. Therefore, to picture a reader, the following 4 criteria can be used. 
		\begin{itemize}
			\item Job title
			\item Education
			\item Industry
			\item Level of interest
		\end{itemize}
	\end{frame}
}

{
	\begin{frame}{3. "Technicalese"}
		It is the pompous, overblown style that leaves makes your writing sound like written by a computer instead of a human being. 
		
		\begin{alertblock}{What gives rise to "technicalese"}
			\begin{itemize}[<+- | alert@+>]
				\item jargon
				\item clich\'{e}s
				\item antiquated phrases
				\item passive sentences 
				\item excess adjectives
			\end{itemize} 
		\end{alertblock}
	\end{frame}
}
{
	\begin{frame}{Avoiding "technicalese"}
		\begin{center}
			\begin{alertblock}{
					\bfseries{Write to express rather than to impress\\}}
			\end{alertblock}
		\end{center}
		
		Other solutions include:
		\begin{description}
			\item[Avoid jargon] Do not use a technical term unless it communicates your meaning precisely. This makes your writing justified amongst non-professionals as well
			\item[Use contradictions] Avoid using clich\'{e}s and antiquated phrases
			\item[Use Active voice] description
		\end{description}
	\end{frame}
}
{
	\begin{frame}{4. Lengthy sentences}
		Lengthy sentences tire the reader and make your writing hard to read. Writing clarity also adds to the style of writing.\\ 
		 
		\begin{alertblock}{Fog Index}
			One measure of writing clarity is the Fog Index, which takes into the sentence and word length. It corresponds to the years of schooling reader needs to be able to understand. It works by, 
			\begin{enumerate}[<+- | alert@+>]
				\item Determine the average sentence length in writing sample. Count each part as a sentence when separated by a semicolon
				\item Calculate the number of big words per 100 words of sample. Not including capitalized words, combinations of short words, or words that are three syllables
				\item Finally, add the average sentence sentence length to the number of big words per 100 words and times 0.4
			\end{enumerate} 
		\end{alertblock}
	\end{frame}
}
{
	\begin{frame}{How to say NO to lengthy sentences?}
		\begin{itemize}
			\item For the Fog Index, a score of 8-9 indicates high-school level; 13, a college freshman; 17, a college graduate. Higher the Fog Index, more difficult the writing. If you score in the upper teens or higher, trim the sentence lengths
			\item Break long sentences into short ones
			\item Use sentence fragments as short sentences are easier to grasp than long ones 
		\end{itemize}
	\end{frame}
}

{
	\begin{frame}{5. Big Words}
			Fancy language frustrates the reader. Big, important-sounding words are very un-technical. \\ 
			Always write in plain and simple English for your readers to understand at level you want them to. Below is a table showing a few errors that occur with alternatives.
			\begin{table}
				\caption{Using big words instead of short ones}
				\begin{tabular}{lr}
					\toprule
					Big Word & Shorter Alternative\\
					\midrule
					Advantageous & Helpful\\
					Facilitate & Help\\
					Ameliorate & Improve\\
					Implement & Carry out\\
					\bottomrule
				\end{tabular}
			\end{table}		
	\end{frame}
}
{
	\begin{frame}{How to avoid Big words}
		Following are some tips:
		\begin{itemize}
			\item Use legitimate technical terms when they communicate your ideas precisely. Do not make words sound impressive 
			\item Technical readers are always interested in detailed information
			\item Keep your writing simple. Provide clarity 
		\end{itemize}
	\end{frame}
}
{
	\begin{frame}{6. What is Writer's Block?}
		It is the inability to start putting words together, caused due to fear anxiety. People are afraid to make mistakes and therefore they review and edit themselves. Professional writing is a result of numerous drafts, edits, deletions and revisions, sometimes by a reader as well. 
		\begin{alertblock}{Tips for improving:}
			\begin{itemize}[<+- | alert@+>]
				\item Break the writing into short sections
				\item Write the easy sections first, like the Methodology instead of Analysis
				\item Write abstracts, introductions, and summaries last
				\item Avoid grammar-book rules
				\item Give some time for thought processing in between your writing. But keep revising
			\end{itemize}
		\end{alertblock}
	\end{frame}
}

{
	\begin{frame}{6. What is Writer's Block?}
		It is the inability to start putting words together, caused due to fear anxiety. People are afraid to make mistakes and therefore they review and edit themselves. Professional writing is a result of numerous drafts, edits, deletions and revisions, sometimes by a reader as well. 
		\begin{alertblock}{Tips for improving:}
			\begin{itemize}[<+- | alert@+>]
				\item Break the writing into short sections
				\item Write the easy sections first, like the Methodology instead of Analysis
				\item Write abstracts, introductions, and summaries last
				\item Avoid grammar-book rules
				\item Give some time for thought processing in between your writing. But keep revising
			\end{itemize}
		\end{alertblock}
	\end{frame}
}
{
	\begin{frame}{7. Poorly defined topic}
		Effective technical writing begins with a clear definition of the specific topic of the paper. Never write about a broad topic that is difficult to define to the reader. \\
		\vspace{5mm}
		A well defined topic also underlines the purpose of the paper. The reader will be very pleased to know what the paper is about before beginning to read. It is very frustrating to through a paper with a vague topic.  
	\end{frame}
}
{
	\begin{frame}{8. Inadequate content}
		Researching a gathering information about the topic is crucial for the purpose of the paper. Average engineers rely on experience rather than information and data when writing a technical paper. What you know is verified through research, digging up facts and data.		
	\end{frame}
}
{
	\begin{frame}{9. Stopping after first draft}
		Many people, when writing reports or journals, stop after producing the first draft. They think its good enough or it passes certain criteria. \\You should start by producing a draft that is just information spitted out on the paper, not worrying about the structure or grammar. The next step is where identify mistakes, rewrite wordy sentences, replace words and fit as per your set outline. Repeat the process again and again. It will only improve the paper. 	
	\end{frame}
}
{
	\begin{frame}{10. Inconsistent usage}
		Inconsistent writing confuses the reader. Your work appears sloppy and unorganized, with no credibility. \\
		
		Good technical writers are consistent with the use of:
		\begin{multicols}{2}
		\begin{itemize}
			\item numbers
			\item hyphens
			\item units of measure
			\item equations
			\item grammar
			\item symbols
			\item capitalization 
			\item technical jargon
			\item abbreviations 
		\end{itemize}	
		\end{multicols}
	\end{frame}
}
{
	\begin{frame}{11. Dull, wordy prose}
		Technical writing is concise. Many engineers write with redundancies --- a needless form of wordiness in which a modifier repeats an idea already contained in the word being modified. 
		\begin{table}
			\caption{Avoiding redundancies}
			\begin{tabular}{lr}
				\toprule
				Redundant & Concise\\
				\midrule
				Actual fact & Fact\\
				Plan ahead & Plan\\
				Still remains & Remains\\
				Written down & Noted\\
				\bottomrule
			\end{tabular}
		\end{table}	
	\end{frame}
}
{
	\begin{frame}{12. Poor page layout}
		Poor page layout like long sections of text disrupt readability. They intimidate the reader. \\
		Use short paragraphs or break long sections to enhance readability. \\
		Moreover, pictures, visuals or drawings attract readers attention. Do not bore the reader with just sections of text. Include relevant pictures for effective communication.
	\end{frame}
}

\section{Conclusion}

\begin{frame}{In the end......}
	Above tips will help you write technical papers without hesitation. Use them to the fullest. \\
	\vspace{10mm}
	\begin{alertblock}{
	%	\begin{center}
			DO NOT FORGET TO PUT SOME ATTITUDE INTO YOUR WRITING! :)}
	%	\end{center}
	\end{alertblock}
	

\end{frame}


\end{document}
